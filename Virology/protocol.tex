%!TEX root=./virology.tex

\section{Lab Protocols}

\subsection{Location of Common Materials}

\begin{tabular*}{\textwidth}{r | p{2in} p{2in}}
\hline
Item & Location & Storage \\
\hline
(In)complete Medium 199 & Incomplete Medium 199 is located in the cold room in the hallway immediately outside the lab. & Complete and serum-free Medium 199s are stored in the $4^{\circ}$C freezer in R2048.\\
Penicillin/Streptomycin Stock & P/S stock is located in the $-20^{\circ}$C freezer in the hallway immediately outside the lab. & Leftover stock is stored in the $4^{\circ}$C freezer in R2048.\\
L-Glutamine Stock & Stock is located in the $-20^{\circ}$C freezer in the hallway immediately outside the lab. & Leftover stock is stored in the $4^{\circ}$C freezer in R2048.\\
Amphotericin B stock & Stock is located in the $-20^{\circ}$C freezer in the hallway immediately outside the lab. & Leftover stock is stored in the $4^{\circ}$C freezer in R2048.\\
0.05\% Trypsin-EDTA & Stock is located in the $-20^{\circ}$C freezer in the hallway immediately outside the lab. & Leftover stock is stored in the $4^{\circ}$C freezer in R2048.\\
Fetal Bovine Serum & Stock is located in the $-20^{\circ}$C freezer in the hallway immediately outside the lab. & Leftover stock is stored in the $4^{\circ}$C freezer in R2048.\\
(In)complete 2x EMEM & Stock is located in the cold room in the hallway immediately outside the lab. & Complete EMEM stock is stored in the $4^{\circ}$C freezer in R2048.\\
SeaPlaque Agarose & Stock is located in a jar above the benches immediately outside of the biosafety cabinet. & Leftover stock should be replaced where you found it.\\
\hline
\end{tabular*}

\subsection{Recording Work and Labelling Materials}

All solutions should be labeled with your name, the date of preparation, and what the solution contains. All cell flasks should be labeled with your name, the date of preparation, the type of cell contained, and the passage of cell contained.

All work completed in the lab should be recorded in a laboratory notebook. Each page should contain entries from only a single day. The date of entry should be recorded at the top of each page. All writing must be easily readable and written in pen. Any errors should be struck through with a single solid line with the correction appearing next to it. Any space on a page not used at the end of a work day should be clearly crossed out in pen.

\subsection{Preparation of Medium 199 (Serum-Free)}

{\bfseries Items Needed:}\footnote{The paper {\itshape Culturing, Storage, and Quantification of Rotaviruses} advises using different quantities of some materials below. Nonetheless, the following are the recommended quantities for use in lab.} \begin{enumerate}
	\item Incomplete Medium 199 ($500$mL)
	\item Penicillin/streptomycin stock ($5$mL)
	\item Amphotericin B stock ($1$mL; $250\mu$g/mL)
\end{enumerate}

In the biosafety cabinet, supplement $500$mL incomplete Medium 199 with $5$mL P/S stock and $1$mL amphotericin B stock. Store at $4^{\circ}$C for up to 3 months.

\subsection{Preparation of Medium 199 (Complete)}

{\bfseries Items Needed:}\footnote{See Note 1.} \begin{enumerate}
	\item Incomplete Medium 199 ($500$mL)
	\item Penicillin/streptomycin stock ($5$mL)
	\item Amphotericin B stock ($1$mL; $250\mu$g/mL)
	\item Fetal bovine serum ($55$mL)
\end{enumerate}

In the biosafety cabinet, supplement $500$mL incomplete Medium 199 with $5$mL P/S stock, $1$mL amphotericin B stock, and $55$mL fetal bovine serum. Store at $4^{\circ}$C for up to 3 months.

\subsection{Preparation of 1.2\% Agarose}

{\bfseries Items Needed:} \begin{enumerate}
	\item SeaPlaque agarose
	\item Milli-Q filtered water
\end{enumerate}

To a $500$mL flask add $1.2$g agarose for every $100$mL water. Inadvisable to fill flask to more than $400$mL. Cap and shake. Loosen lid. Apply autoclave tape to the lid. Autoclave approx. 20 min.

\subsection{Preparation of 2x EMEM (Serum-Free)}

{\bfseries Items Needed:} \begin{enumerate}
	\item incomplete 2x EMEM ($500$mL)
	\item $200$mM \textsc{l}-glutamine ($10$mL)
	\item P/S stock ($10$mL)
	\item $250\mu$g/ml amphotericin B stock ($1$mL)
\end{enumerate}

In the biosafety cabinet, to incomplete 2x EMEM stock add \textsc{L}-glutamine, P/S stock, and amphotericin B stock. Store at $4^{\circ}$C for up to 3 months.

\subsection{Preparation of PBS (Phosphate Buffered Saline)}

\subsection{Procedure for Splitting MA104 Cells}

{\bfseries Items Needed:} \begin{enumerate}
	\item 1x PBS
	\item 0.05\% Trypsin-EDTA
	\item Complete Medium 199
	\item $150$cm$^2$ flask
\end{enumerate}

In a water bath, warm PBS, Trypsin, and complete Medium 199 to $37^{\circ}$C. Transfer all materials into the biosafety cabinet. Tilt the flask with your cells (having formed a confluent monolayer) such that the culture medium collects on the sloped surface leading to the neck of the flask. Vacuum out all culture medium using a small glass pipet inserted into the rubber vacuum hose. To the cell culture, add $5$mL Trypsin, tilting the flask forward and back to ensure that the cells are fully bathed. Vacuum out the Trypsin.

Add a second $5$mL portion of Trypsin to the cell culture. Incubate at $37^{\circ}$C until all cells have detached from the surface and are free-floating. Lightly tap the flask with your hand if any cells remain attached. Add $15$mL complete Medium 199. (If you used less trypsin in the previous step, adjust the amount of Medium 199 applied such that there is $20$mL solution in the flask.)

To each new flask that you wish to prepare, add complete Medium 199 such that the final flask volume after the cell mixture is added will be equal to $25$mL. To determine the volume of cell mixture to add to each new flask, we use Table 1.1.

\begin{figure}[htp]
{\bfseries Table 1.1}\\[0.1cm]
\begin{tabular*}{\textwidth}{c c}
\hline
Cell dilution ratio & Cell mix volume to add \\
\hline
1:2 & 10mL \\
1:4 & 5mL \\
1:8 & 2.5mL \\
\hline
\end{tabular*}\\[0.1cm]
{\small {\itshape Note.} For example, if we wished to prepare one 1:4 dilution and two 1:8 dilutions, to our first flask we would add 15mL complete Medium 199 and 10mL cell mixture; to each of our second two flasks we would add 20mL complete Medium 199 and 5mL cell mixture.}
\end{figure}

Cap the new flask(s) and tilt forward and back to evenly spread the cells. Loosen the lids and incubate at $37^{\circ}$C.

\subsection{Procedure for Plating MA104 Cells}

\subsection{Procedure for activating RV SA11 and Infecting MA104 Cells}

\subsection{Procedure for Performing RV Plaque Assay}

\subsection{Procedure for Transducing MA104 Cells with siRNA-Expressing Lentiviral Vectors}