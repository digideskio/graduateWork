\documentclass[11pt,final] {article}
\usepackage[margin=1in]{geometry}
\usepackage{setspace}
\usepackage{amsmath}
\usepackage{amsfonts}
\usepackage{fancyhdr}
\usepackage{multicol}
\usepackage{graphicx}

\pagestyle{fancy}
\renewcommand{\headrulewidth}{0pt}
\fancyhead{}
\fancyfoot{}
\rhead{\thepage}

\linespread{2}

% \title{The Title}
% \author{The Author}
% \date{\today}

\begin{document}
% \maketitle

\noindent Christopher Wetherill\\
Week 07 --- Addiction\\
Abstinence project\\[0.25cm]

For this assignment I abstained from consumption of coffee for the duration of the week. I will typically drink the equivalent of 1 pot of coffee or more on any given day. Given my physiological dependence on caffeine, I opted to only abstain from coffee, but still allow for caffeinated tea to prevent the onset of any withdrawal-induced migraines. On a 100-point scale, I rate coffee, let's arbitrarily say, 87. Although this score is partly due to my dependence on caffeine, the act of drinking coffee is also just relaxing and enjoyable.

With respect to the clinical course, I did not employ any coping strategies beyond simply not drinking coffee (and having a cup of tea whenever I felt a headache coming on, though not in excess of two cups of tea per day). Although I've developed a physiological dependence on caffeine and subjectively enjoy drinking coffee, I apparently have not developed a psychological addiction to it: I did not crave it; I did not relapse. It took me a couple extra hours each morning to feel fully alert, but otherwise observed no significant physical or emotional changes.

I am, however, skeptical that this strategy would be effective among individuals with substance abuse disorders. Specifically, as far as I know, I do not have deficits in prefrontal cortical or limbic/paralimbic functioning (and correspondingly the decision-making process and long-term planning and delay of gratification) that may be characteristic of individuals with SUDs; even with physiological dependence, I don't make a very good addict (I do 3 month-long caffeine detoxes each year, for instance): just stopping substance use out of the blue for no good reason is typically not a successful strategy.

\end{document}