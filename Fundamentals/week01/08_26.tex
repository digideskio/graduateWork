%!TEX root=../notes.tex

%%%%%%%%%%%%%%%%%%%
% Grand challenge: Cancer.
%%%%%%%%%%%%%%%%%%%

\section{Fragile X Syndrome}
\index{Topic!Neuroscience}

\subsection{Overview}
Fragile X Syndrome (also known as Martin-Bell syndrome, or Escalante's syndrome) is a sex-linked genetic disorder causing mental retardation and one of the largest single-gene causes of autism among children.

Specifically, Fragile X Syndrome (FX) is most often caused by a simple CGG repeat in the FMR1 gene on the X chromosome. This, in turn, causes an underexpression of the Fragile X mental retardation protein (FMRP\index{Fragile X mental retardation protein}). The amount of FRMP present is demonstrated to affect synaptic plasticity (with an abundance correlating to increased plasticity), and thereby to affect learning. As such, when the protein is absent, there is reduced plasticity in the individual, yielding mental retardation. The degree of retardation is inversely proportional to the amount of FMRP present in the individual.

Moreover, research has indicated in mouse models that there is a functional opposition between FMRP and certain metabotropic glutamate receptors (mGluRs\index{Metabotropic glutamate receptors}), and specifically mGluR5. FMRP and mGluR5 both appear to regulate mRNA translation at the synapse, with mGluR5 inhibiting translation and FMRP promoting it. As such, in the absence of FMRP, mRNA translation at the synapse is highly downregulated, resulting in the observed mental retardation.

In neonates, long-term potentiation (LTP\index{Long-term potentiation}) appears to be important for retaining newly-formed synapses and long-term depression (LTD\index{Long-term depression}) for activity-guided synapse elimination. Further, postsynaptic group 1 mGluR activation is responsible in many parts of the brain for LTD. However, for this to occur, there must be a rapid translation of pre\"{e}xisting mRNA in the postsynaptic dendrites. Given that FMRP mRNA is found in dendrites and that FMRP binds mRNA, it is perhaps unsurprising that in the absence of FMRP (in \textit{Fmr1} knockout mice), exaggerated protein-synthesis-dependent LTD has been observed. 

``According to this model, mGluR activation normally stimulates synthesis of proteins involved in stabilization of LTD and, in addition, FMRP. The FMRP functions to inhibit further synthesis (an example of end-product inhibition), and puts a brake on LTD'' (Bear, Huber, \& Warren). That is, in a typical individual, FMRP acts as a negative feedback to LTD, thus promoting normal growth. When it is absent, however, LTD is allowed to continue unimpeded, resulting in excessive activity-guided synapse elimination. This interaction between FMRP and mGluRs may thus result both in an insufficient number of synaptic interconnections at birth (via synaptic pruning) and an inability to effectively form these connections throughout life (via reduced synaptic plasticity resulting from mGluR-induced LTD).