%!TEX root=../notes.tex

%%%%%%%%%%%%%%%%%%%
% Grand challenge: Cancer.
%%%%%%%%%%%%%%%%%%%

\section{TBMH Fundamentals Overview}

\index{Topic!Overview}For the purpose of this course and program, translational biology is the act or process of changing some aspects of knowledge about living organisms and vital processes. Notably, the translational element of this program does not specifically apply to either medicine or health.

\subsection{Translational Research}
\index{Translational research}Although the definition and stages of translational research vary, we will define it in five stages, broadly, as:

\begin{enumerate}
	\item \textbf{T0} --- The focus is on identifying problems, opportunities, and novel approaches to tackle those problems and opportunities. This is characterizes by the identification of approaches to relevant health problems with research beginning with a basic research question.
	
	\item \textbf{T1} --- Discovery or foundational research and development of treatments. This is translational research in which findings are moved from basic research to testing for clinical effect and/or applicability.
	
	\item \textbf{T2} --- Health application to assess efficacy of treatment. This research assesses the value of the treatment in a clinical trial (i.e., in Phase III clinical trials).
	
	\item \textbf{T3} --- Health practice; science of dissemination and implementation. This attempts to move evidence-based guidelines into health practice (analogous to Phase IV clinical trials).
	
	\item \textbf{T4} --- Evaluation of health impact on real-world populations. This is, broadly, outcomes research that monitors the morbidity, mortality, benefits, and risks in the population of the treatment.
\end{enumerate}

Specifically, this is an iterative process with each stage building on and informing every other.

\subsection{Clinical Trials}

The phases of clinical trials are as follow (via the NIH):
\begin{enumerate}
	\item \textbf{Phase I} --- Researchers test a new drug or treatment in a small group for the first time to evaluate its safety, determine its dosage range, and identify side effects.
	
	\item \textbf{Phase II} --- The drug or treatment is given to a larger group of people to see if it is effective and to further evaluate its safety.
	
	\item \textbf{Phase III} --- The drug or treatment is given to large groups of people to confirm its effectiveness, monitor side effects, compare it to commonly-used treatments, and collect information that will allow the drug to be used safely.
	
	\item \textbf{Phase IV} --- Studies done after the drug is places on the market to gather information on its effect in varied populations and any side effects associated with long-term use.
\end{enumerate}