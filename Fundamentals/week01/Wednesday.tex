%!TEX root=../notes.tex

\section{Age-Related Macular Degeneration}
\index{Topic!Development, Aging, and Repair}

\subsection{Overview}

Possession of one or more chronic disease or disorder has been found (Moussavi, 2007) to significantly correlate with an elevated prevalence of depression. Significantly higher likelihood for comorbid depression is found among those with multiple chronic disorders. Moreover, visually-limiting eye disease specifically is significantly correlated with an increased risk of comorbid depression among adults (Popescu et al., 2012). Among those with visually-limiting eye disease, an integrated mental health and low vision intervention may significantly reduce the risk of comorbid depression (Rovner et al., 2014).

Age-related macular degeneration (AMD)\index{Age-related macular degeneration} is an age-related disease characterized by the loss of vision in the center of the visual field (i.e., the macula lutea). It has both dry and wet forms. In the dry form, cellular debris accumulates between the retina and the choroid and may cause the retina to detach. In the wet form, blood vessels in the choroid grow abnormally and occlude vision. ``The disease pathology emerges with the degeneration of macula which forms the central part of retina. The macula consists of photoreceptor (rods and cones) important for central vision'' (Sharma, Sharma, \& Anand, 2014).

Several age-related cellular processes have been identified as potential disruptors to ocular homeostasis, including:

\begin{enumerate}
	\item metabolic pathways (Uchiki et al., 2012);
	\item telomere shortening;
	\item impaired mechanism of autophagy;
	\item disrupted proteolytic and lysosomal function (Viiri et al., 2013);
	\item decline in ability to combat oxidative stress (Cutler et al., 2004); and
	\item enhanced mitochondrial dysfunction.
\end{enumerate}

The emergence of AMD arises from a complex interaction of both environmental (e.g., smoking, IV-B exposure) and genetic (e.g., immune system components, angiogenic factors) that is not yet fully understood. ``Clinical trials have shown that VEGF [vascular endothelial growth factor] antagonists provide major benefits for patients with subretinal NV [neovascularization] due to AMD and even greater benefits are seen by combining antagonists of VEGF and PDGF-B [platelet-derived growth factor-B]'' (Campochiaro, 2013). Additionally, \textit{in vivo} research (non-viral gene therapy) by Iriyama et al. (2011) has shown promise for reversing the effects of choroidal neovascularization by transfection of soluble fms-like tyrosine kinase-1 (sFlt-1) with the polyion complex (PIC) micelle encapsulating DNA.