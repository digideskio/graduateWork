%!TEX root=./labNotes.tex

\section{Location of Common Materials}

\begin{tabular*}{\textwidth}{r | p{2in} p{2in}}
\hline
Item & Location & Storage \\
\hline
\end{tabular*}

\section{Recording Work and Labeling Materials}

All solutions should be labeled with your name, the date of preparation, and what the solution contains. All cell flasks, plated cells, etc. should be labeled with your name, the date of preparation, the type of cell contained, the passage of cell contained, and, if applicable, any agents with which the cell has been transduced or infected.

All work completed in the lab should be recorded in a laboratory notebook. Each page should contain entries from only a single day if handwritten. The date of entry should be recorded at the top of each page. All writing must be easily readable and written in pen. Any errors should be struck through with a single solid line with the correction appearing next to it. Any space on a page not used at the end of a work day should be clearly crossed out in pen.

If an electronic lab notebook is used, the same general guidelines apply; however, multiple entries may appear on a single printed page. There must additionally be maintained an audit log of all changes and alterations made to the notebook to which the author has read-only access (i.e., the author must be unable to tamper with the audit log).

\section{Procedures for Autoclaving}

Ensure that all items have autoclave tape (if necessary). Place all items into a Nalgene tub. Take to the autoclave room. Add an autoclave quality indicator strip to the bin and insert into the autoclave. Close the door and select the appropriate options based on what is being sterilized. Start the cycle and fill out the autoclave use form on the bench next to the machine. Once the cycle is complete, retrieve the bin using the thick insulated gloves found in the lab (next to where the bin is stored). It is normal for there to be a small amount of water in the base of the bin.

\section{Preparation of Co-Cultured MDCK and 293T Cells}

{\bfseries Items Needed:}
\begin{enumerate}
	\item Serum-free DMEM
	\item Serum-free MEM
	\item Fetal Bovine Serum
	\item Penicillin/streptomycin stock
	\item Phosphate-buffered saline
	\item $0.05\%$ Trypsin-EDTA
\end{enumerate}

In the biosafety cabinet aspirate the cell culture medium from each flask. Wash each flask with $10$mL PBS and aspirate. Wash the MDCK cells with $5$mL trypsin and aspirate. Add $5$mL trypsin to the MDCK flask and incubate for several minutes until cells detach.

While waiting for MDCK cells to detach, wash 293T cells in $3$mL trypsin and aspirate. Add $3$mL trypsin to 293T cells and incubate until cells detach. Add $1$mL FBS, $9$mL DMEM stock and pipet until a single-cell suspension has been formed.

To the detached MDCK cells add $3$mL trypsin, $8$mL MEM and transfer in equal portions to two Eppie tubes for centrifugation. Spin cell solutions at $1000\times$RPM for $5$ minutes at $21^{\circ}$C. Return MDCK cells to the flask and pipet until a single-cell suspension has been formed.

Perform a cell count for each cell population. To do this, obtain a cell counter slide, label each well with the cell population being used, and add a $20\mu$L aliquot of the cell solution to the appropriate well. Using the Cellcount Vision software, count each population of cells using a concentration assay.

For plating in one $6$-well plate, a total of $6$ million cells are needed at $1$ million cells per well with a $70:30$ ratio of 293T:MDCK cells. Each well should receive $2$mL solution.

For example, given concentrations of $3.25\e{6}$ 293T cells per well and $1.5\e{6}$ MDCK cells per well, we would need:

\begin{eqnarray*}
%
% 293 T Cells
%
% Cell final number
\text{293T cells} &=& 6\e{6}\text{ cells} \times 0.70 \\
&=& 4.2\e{6}\text{ cells} \\
% Cell equivalent volume
\text{Volume 293T cells} &=& \frac{4.2\e{6}\text{ cells} \times 1\text{mL}}{3.25\e{6}\text{ cells}} \\
&=& 1.29\text{mL 293T cell solution} \\
%
% MDCK Cells
%
% Cell final number
\text{MDCK cells} &=& 6\e{6}\text{ cells} \times 0.30 \\
% Cell equivalent volume
&=& 1.8\e{6}\text{ cells} \\
\text{Volume MDCK cells} &=& \frac{1.8\e{6}\text{ cells} \times 1\text{mL}}{1.5\e{6}\text{ cells}} \\
&=& 1.2\text{mL MDCK cell solution} \\
\end{eqnarray*}

We then supplement the combined cell solution with DMEM to reach a total volume of $12$mL for a distribution of $2$mL solution per well. To each well add $2$mL combined cell solution and spread evenly in each well be shaking. Label plate and incubate at $37^{circ}$C overnight.

Supplement flasks as needed with (D)MEM and P/S stock.