\documentclass[11pt,letterpaper,final] {article}

%%%%%%%%%%%%%%%%%%%%%%%%%%%%%%
% Packages
%%%%%%%%%%%%%%%%%%%%%%%%%%%%%%

	\usepackage[margin=1in]{geometry}
	\usepackage{amsmath}
	\usepackage{amsfonts}
	\usepackage{fancyhdr}
	\usepackage{graphicx}
	% \usepackage{tikz}
	% \usepackage{setspace}
	% \usepackage{multicol}

%%%%%%%%%%%%%%%%%%%%%%%%%%%%%%
% Page styling
%%%%%%%%%%%%%%%%%%%%%%%%%%%%%%
	
	%%%%%%%%%%%%%%%%%%%%
	% Headers and footers
	%%%%%%%%%%%%%%%%%%%%
	
	\pagestyle{fancy}
	\renewcommand{\headrulewidth}{0pt}
	\fancyhead{}
	\fancyfoot{}
	\rhead{\thepage}
	\lhead{Christopher Wetherill}
	
	%%%%%%%%%%%%%%%%%%%%
	% Graphics path
	%%%%%%%%%%%%%%%%%%%%
	
	%\graphicspath{{./assets/}}
	
	%%%%%%%%%%%%%%%%%%%%
	% Frontmatter
	%%%%%%%%%%%%%%%%%%%%
	
	% \title{The Title}
	% \author{The Author}
	% \date{\today}
	
	

%%%%%%%%%%%%%%%%%%%%%%%%%%%%%%
% Custom definitions
%%%%%%%%%%%%%%%%%%%%%%%%%%%%%%
	% Easy scientific notation
	\newcommand{\e}[1]{\ensuremath{\times 10^{#1}}}
	
	% Textual subscripts
	\newcommand{\sub}[1]{\ensuremath{_{\text{#1}}}}
	
	% Textual superscripts
	\newcommand{\super}[1]{\ensuremath{^{\text{#1}}}}

\begin{document}
% \maketitle

\noindent {\bfseries Seminar:}\\
Dr. Kirk Knowlton, University of California, San Diego\\
The Myocardial Cytoskeleton and Intercalated Disc: A Role in Cardiomyopathy and Conduction Abnormalities\\
Thursday, March 19; 5:30 P.M. -- 6:30 P.M.\\
VTCRI Distinguished Scholars Series\\[0.4cm]

\noindent {\bfseries Reason for Attendance:}

The focus of the talk was viral myocarditis --- viral infection of heart tissue that occurs in spite of the fact that universally viruses are not optimized to infect or replicate in cardiac tissue, nor does any tissue-specific virus preferentially localize to the heart. The mechanisms that facilitate this type of viral infection are poorly understood and Dr. Knowlton's work has sought to elucidate some of the mechanisms that allow viruses to localize to and successfully infect these tissues.\\[0.4cm]

\noindent {\bfseries Research Findings:}

Coronary heart diseases represent a relatively well-characterized set of etiologies for heart failure (e.g., high circulating glucose levels; high cholesterol) and successful treatments against these specific causes are also well established. However, understanding of the causes of, and treatments for, cardiomyopathies is considerably less advanced.

Specifically, with respect to dilated cardiomyopathy we understand that heredity, viral, inflammatory, drug-induced, idiopathic, and other factors contribute to its onset. An estimated 30 -- 40\% of cases are thought to have a genetic component, yet only an estimated 40\% of the contributing genetic factors have been clearly elucidated.

Another known cause of dilated cardiomyopathy is viral myocarditis. The exact incidence of this is unknown, largely due to the difficulty in its diagnosis and the range of viruses that can give rise to it --- from flu to rabies to coxsackievirus. In these cases, dilated cardiomyopathy is thought to result from a combination of cardiac damage from viral replication and from the immune response to the virus's presence.

In this context, Dr. Knowlton's research has sought to understand the mechanism of viral infection of cardiac cells and any potential crosstalk between genetic and acquired cardiomyopathy. Specifically, this research has found that viral cardiomyopathy causes cleavage of the dystrophin protein, disrupts the dystrophin-glycoprotein complex, and ultimately increases cellular permeability, facilitating viral egress from the cell.

Following this discovery, his lab asked what would happen in infection of a dystrophin-deficient heart. It was found that viral titers were considerably higher and the pathology much worse. Indeed, this also appears to be the case among patients with Duchenne's muscular dystrophy who suffer from viral myocarditis. Contrastingly, the modification of dystrophin to prevent its cleavage in turn sustains cell membrane stability and results in decreased viral titers, earlier resolution of infection, and decreased inflammation.

\end{document}