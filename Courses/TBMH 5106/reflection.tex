\documentclass[11pt,letterpaper,final] {article}

%%%%%%%%%%%%%%%%%%%%%%%%%%%%%%
% Packages
%%%%%%%%%%%%%%%%%%%%%%%%%%%%%%

	\usepackage[margin=1in]{geometry}
	\usepackage{amsmath}
	\usepackage{amsfonts}
	\usepackage{fancyhdr}
	\usepackage{graphicx}
	% \usepackage{apacite}
	% \usepackage{tikz}
	% \usepackage{setspace}
	% \usepackage{multicol}
	% \usepackage[left]{lineno}

%%%%%%%%%%%%%%%%%%%%%%%%%%%%%%
% Page styling
%%%%%%%%%%%%%%%%%%%%%%%%%%%%%%
	
	%%%%%%%%%%%%%%%%%%%%
	% Headers and footers
	%%%%%%%%%%%%%%%%%%%%
	
	\pagestyle{fancy}
	\renewcommand{\headrulewidth}{0pt}
	\fancyhead{}
	\fancyfoot{}
	\rhead{\thepage}
	
	%%%%%%%%%%%%%%%%%%%%
	% Graphics path
	%%%%%%%%%%%%%%%%%%%%
	
	%\graphicspath{{./assets/}}
	
	%%%%%%%%%%%%%%%%%%%%
	% Frontmatter
	%%%%%%%%%%%%%%%%%%%%
	
	% \title{The Title}
	% \author{The Author}
	% \date{\today}


%%%%%%%%%%%%%%%%%%%%%%%%%%%%%%
% Custom definitions
%%%%%%%%%%%%%%%%%%%%%%%%%%%%%%
	% Easy scientific notation
	\newcommand{\e}[1]{\ensuremath{\times 10^{#1}}}
	
	% Textual subscripts
	\newcommand{\sub}[1]{\ensuremath{_{\text{#1}}}}
	
	% Textual superscripts
	\newcommand{\super}[1]{\ensuremath{^{\text{#1}}}}

\begin{document}

% \linenumbers
% \maketitle

	\noindent Christopher Wetherill\\
	TBMH 5106\\
	Reflection\\[0.4cm]

	\paragraph{Completion of an IDP:} Please note at least one positive thing you gained from the IDP class session, and one way the MyIDP could be improved upon.\\
	
	I was left unsure how I felt about the online IDP. At a glance, I would question whether the skills, interests, and values assessments have been properly validated and how the subsequent ``Career Path Matches'' are ranked as a function of these self-report measures. Although convenient to see a quick tabulation of common careers following a PhD, I don't know that the match results offer any particularly insightful or psychometrically valid advice or information. Certainly, if an individual had never considered his or her skills, interests, and possible career paths before, this tool could be more useful; however, my first guess would be that the intended audience will have generally already given a considerable amount of time and attention to these factors.
	
	If anything, I might like to see more specific and tailored information and testimonials presented with respect to various non-academic career options. The linkout resources that were provided were hit-or-miss: some were genuinely useful and compelling; others a little gimmicky. I understand that it is difficult to strike a balance between a general overview of common career paths and an in-depth look at those specific, individual careers, but for me personally the current balance weights the generalist approach a little too heavily.
	
	\paragraph{Have a discussion with an advisor to set a goal:} Have a discussion with your thesis/rotation advisor, research supervisor, or other mentor to set at least one skill, one project, and one career advancement goal. Did this happen? Was it helpful? What were your goals?\\
	
	I spoke with my rotation mentors to set goals for each rotation in terms of techniques and projects to learn and accomplish, respectively. We also discussed some longer-term career-planning-type things.
	
	\paragraph{Have an informational interview:} Have at least one informational interview (or externship experience) with a faculty or professional in a career that interests you, and be prepared to report on their suggestions for career growth and networking. I’d prefer this person to be someone you sought out on your own. Did this happen? How did you identify the professional? Did you feel differently before/after the experience? Was it helpful? Are you more likely to do it again?\\
	
	I spoke with the Disease Intervention Specialist for the New River and Mt. Rogers Health Districts. Rather than discussing career growth and networking, we talked a little more specifically about the position itself and what work it entails. It was useful: although it isn't a career I would necessarily like to pursue, it introduced new perspectives and a fresh approach to infectious disease and, particularly, their human toll and transmission.
	
	\paragraph{Attend a professional development activity beyond class:} Attend at least one professional development activity, event, or workshop beyond this class. What did you attend and what did you get out of it?\\
	
	I attended a webinar, ``SBIR/STTR for PostDocs \& Grad Students,'' hosted by the Center for Innovative Technology. Previously I had been unaware of such government-sponsored grants meant to encourage tech transfer from universities to start-ups. Certainly, given the general reluctance of most major VCs to provide seed and series A capital to biotech start-ups, it was valuable to learn about other potential early stage sources of funding.
	
	\paragraph{BEST advocacy:} Can you think of ways that BEST students and postdocs can be advocates for the program and professional development in general? Is there more/less you'd like to see from this course?\\
	
	I think, broadly, this course struck a good balance between both personal and professional development and did a good job introducing both academic and non-academic career paths (and illustrating that they can frequently intersect). A couple of the classes were a little nebulous, particularly given how distinctly applied the day's topics were (e.g., communicating science to an audience without your field-specific knowledge); however, by and large I felt that I took away something new and valuable from each class session.

%%%%%%%%%%%%%%%%%%%%
%% REFERENCES
%%%%%%%%%%%%%%%%%%%%

% \clearpage
% \bibliographystyle{apacite}
% \bibliography{references}

\end{document}