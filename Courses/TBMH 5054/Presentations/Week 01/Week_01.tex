\documentclass[11pt,letterpaper,final] {article}

%%%%%%%%%%%%%%%%%%%%%%%%%%%%%%
% Packages
%%%%%%%%%%%%%%%%%%%%%%%%%%%%%%

	\usepackage[margin=1in]{geometry}
	\usepackage{amsmath}
	\usepackage{amsfonts}
	\usepackage{fancyhdr}
	\usepackage{graphicx}
	\usepackage{setspace}
	% \usepackage{tikz}
	% \usepackage{setspace}
	% \usepackage{multicol}

%%%%%%%%%%%%%%%%%%%%%%%%%%%%%%
% Page styling
%%%%%%%%%%%%%%%%%%%%%%%%%%%%%%
	
	%%%%%%%%%%%%%%%%%%%%
	% Headers and footers
	%%%%%%%%%%%%%%%%%%%%
	
	\pagestyle{fancy}
	\renewcommand{\headrulewidth}{0pt}
	\fancyhead{}
	\fancyfoot{}
	\rhead{\thepage}
	\lhead{Wetherill --- TBMH 5054}
	
	%%%%%%%%%%%%%%%%%%%%
	% Graphics path
	%%%%%%%%%%%%%%%%%%%%
	
	%\graphicspath{{./assets/}}
	
	%%%%%%%%%%%%%%%%%%%%
	% Frontmatter
	%%%%%%%%%%%%%%%%%%%%
	
	% \title{The Title}
	% \author{The Author}
	% \date{\today}
	


%%%%%%%%%%%%%%%%%%%%%%%%%%%%%%
% Custom definitions
%%%%%%%%%%%%%%%%%%%%%%%%%%%%%%
	% Easy scientific notation
	\newcommand{\e}[1]{\ensuremath{\times 10^{#1}}}
	
	% Textual subscripts
	\newcommand{\sub}[1]{\ensuremath{_{\text{#1}}}}
	
	% Textual superscripts
	\newcommand{\super}[1]{\ensuremath{^{\text{#1}}}}

\begin{document}
% \maketitle

\doublespacing

\noindent {\bfseries Probiotics for the Modulation of the Microbiome}

Diet is considered a major driver of gut microbiota diversity and has a clear impact on the functional relationship of the gut microbiome with the host. The gut microbiota themselves exert great impact on host's nutritional status; provide additional enzymatic activity; and be modulated by food bioactive compounds, positively or negatively. Given this relationship between what we ingest and the functional status of our gut microbiome, can these interactions be purposefully manipulated via the administration of probiotic and prebiotic compounds?

Although the answer broadly seems to be yes, the reality appears to be slightly more complex. Specifically, probiotics must be alive at administration and able to pass the gastric and ileal environments; few dose-response studies exist to establish the necessary quantity of probiotic needed for an intended effect; it is often difficult to quantitate the beneficial effects in the absence of functional abnormalities in your target population\footnote{i.e., the population taking these is not generally chronically ill and benefits are reported subjectively, not clinically.}; and, importantly, must not contain transferrable antibiotic-resistant genes. Moreover, the evaluation of a probiotic's effects must be limited to a single strain of bacteria or a predefined combination of specific strains.

Despite having seen success using animal models to evaluate the effects of probiotic modulation, we are still only beginning to unravel the molecular mechanisms of probiotic cell modulation. Yet, functional genomic approaches combined with dedicated mutagenesis approaches are helping to elucidate molecular mechanisms of probiotic health effects and pave the way for more cost‐effective and targeted applications of these bioactive probiotic molecules.

\clearpage

\noindent {\bfseries Fungal Pathogens}

Despite having utilized fungi for several thousands of years, relatively little research has been devoted to elucidate the mechanisms by which they evade the host immune response. Broadly, we know that CLEC7A, CARD9, IL-17, and IL-22 have been implicated in host immune clearance of fungi; however, beyond this little is known of how commensal intestinal fungi interact with the host immune system.

Broadly, for a fungus to move from commensal to pathogenic, we see the need for 3 factors to be satisfied: that there is some pre{\"e}xisting gut microbial dysregulation; that there is increased permeability of the intestinal mucosal barrier; and that the host is somehow immunocompromised.

For instance, {\itshape Candida albicans}, a yeast that exists as a commensal organism in the oral cavities and GI tracts of about $80$\% of the human population, only opportunistically becomes pathogenic when the host's immune system becomes otherwise compromised. Once this occurs, {\itshape C. albicans} likely uses transcriptional programs to invade tissue and further evade immune response. However, {\itshape in vivo} transcriptional profiling may be needed to describe the progression of infection and
complex time-dependent patterns of gene expression.

In addition to transcriptional programs, some fungi have been seen to use adaptive sexual reproduction to evade the host immune response. Historically, fungi have been thought to reproduce either asexually/parasexually (typically pathogenic fungi) or sexually (commensal fungi). However, recent evidence suggests that this dichotomy is slightly more flexible than this. For instance, {\itshape C. neoformans}, although it typically reproduces through mitosis, can undergo a meiotic process that promotes recombinational repair in the host’s oxidative environment and is able to directly exchange genetic information with non-isogenic $\alpha$-strains.

\clearpage
\noindent {\bfseries Obesity and the Microbiome}

Via twin studies, we have found that there exists a statistically significant link between gut microbiome genetic diversity and obese status, specifically with obese individuals having decreased genetic diversity of their gut microbiomes. Moreover, however, these studies have also revealed a functional core that a healthy microbiome constitutes, regardless of the specific combination of gut microbes. Among obese individuals, this functional core becomes shifted to promote systems involved in carbohydrate processing.

We further see that diet has a direct and relatively immediate impact on gut microbiome composition and health. Not only do high-fat diets see a decrease in {\itshape Bacteroidetes} and increases in {\itshape Firmicutes} and {\itshape Proteobacteria}, but microbiotal responses to dietary changes also differ depending on host lean/obese status. Specifically, it appears to be much easier to deplete gut levels of {\itshape Bacteroidetes} in lean individuals than it is to replenish them and deplete gut levels of {\itshape Firmicutes} in obese individuals.

Yet, these studies have only addressed the correlation between gut microbiome composition and obese status. For a causal relationship, we can look to several animal models. For instance, a recent study administered {\itshape E. coli} engineered to express NAPEs---a precursor to a family of lipids that promote feelings of satiety---and found overall reductions in weight gain and fat mass sustained beyond the period of administration. In a separate study, researchers found that inducible epithelial-cell specific deletion of MyD88 not only inoculates against diet-induced obesity, but moreover that these protective effects can be transferred following gut microbiota transplantation, leaving obvious implications for targeted fecal transplant procedures.

\clearpage

\begin{center}
References
\end{center}

\singlespacing
\renewcommand{\labelenumi}{[\arabic{enumi}]}
\linespread{1}

\begin{enumerate}
\item Araya, M., Morelli, L., Reid, G., Sanders, M., \& Stanton, C. (2002). Guidelines for the evaluation of probiotics in food. Jt. FAO/WHO Work. Group Rep., London, Ontario, Can. ftp://ftp.fao.org/es/esn/food/wgreport2.pdf
\item Aureli, P., Capurso, L., Castellazzi, A., Clerici, M., Giovanni, M., Morelli, L., et al. (2011). Probiotics and health: An evidence-based review. Pharmacological Research, 63, 366 -- 376.
\item Kleerebezem, M., \& Vaughan, E. (2009). Probiotic and gut lactobacilli and bifidobacteria: Molecular approaches to study diversity and activity. Ann. Rev. Microbiol., 63, 269 -- 290.
\item Laparra, J., \& Sanz, Y. (2010). Interactions of gut microbiota with functional food components and nutraceuticals. Pharmacological Research, 61, 219 -- 225.
Lee, I., Tomita, S., Kleerebezem, M., \& Bron, P. (2013). The quest for probiotic effector molecules: Unraveling strain specificity at the molecular level. Pharmacological Research, 69, 61 -- 74.
\item Martin, F., Wang, Y., Sprenger, N., Yap, I., Lundstedt, T., Lek, P., et al. (2008). Probiotic modulation of symbiotic gut microbial-host metabolic interactions in a humanized microbiome mouse model. Molecular Systems Biology, 4:157.
\item Sanz, Y., Rastmanesh, R., \& Agostonic, C. (2013). Understanding the role of gut microbes and probiotics in obesity: How far are we? Pharmacological Research, 69, 144 -- 155.

\item Berg, R. (1999). Bacterial translocation from the gastrointestinal tract. Advanced Experimental Medical Biology, 473, 11 -- 30.
\item Hube, B. (2004). From commensal to pathogen: Stage- and tissue-specific gene expression of Candida albicans. Current Opinion in Microbiology, 7, 336 -- 341.
\item Lin, X., Hull, C., \& Heitman, J. (2005). Sexual reproduction between partners of the same mating type in Cryptococcus neoformans. Nature, 434, 1017 -- 1021.
\item Michod, R., Bernstein, H., \& Nedelcu, A. (2008). Adaptive value of sex in microbial pathogens. Infection, Genetics, and Evolution, 8, 267 -- 285.
\item Nielsen, K., \& Heitman, J. (2007). Sex and virulence of human pathogenic fungi. Advances in Genetics, 57, 143 -- 173.
\item Shoham S, Levitz SM. (2005). The immune response to fungal infections. British Journal of Haematology, 129, 569 -- 582.
\item Underhill, D., \& Iliev, I. (2014). The mycobiota: interactions between commensal fungi and the host immune system. Nature Reviews Immunology, 14, 405 -- 416.

\item Chen, Z., Guo, L., Zhang, Y., et al. (2014). Incorporation of therapeutically modified bacteria into gut microbiota inhibits obesity. Journal of Clinical Investigation, 124, 3391 -- 3406.
\item DiBaise, J., Frank, D., \& Mathur, R. (2012). Impact of the gut microbiota on the development of obesity: Current concepts. American Journal of Gastroenterology, 1, 22 --27.
\item Everard, A., Geurts, L., Caesar, R., et al. (2014). Intestinal epithelial MyD88 is a sensor switching host metabolism towards obesity according to nutritional status. Nature Communications, 5:5648.
\item Fleissner, C., Huebel, N., Abd El-Bary, M., et al. (2010). Absence of intestinal microbiota does not protect mice from diet-induced obesity. British Journal of Nutrition, 104, 919 --929.
\item Hildebrandt, M., Hoffmann, C., Sherrill-Mix, S., et al. (2009). High-fat diet determines the composition of the murine gut microbiome independently of obesity. Gastroenterology, 137, 1716 --1724.
\item Turnbaugh, R., Hamady, M., Yatsunenko, T., et al. (2009). A core gut microbiome in obese and lean twins. Nature, 457, 480 -- 485.
\end{enumerate}
\end{document}