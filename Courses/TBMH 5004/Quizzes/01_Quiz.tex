\documentclass[11pt,final] {article}
\usepackage[margin=1in]{geometry}
\usepackage{setspace}
\usepackage{amsmath}
\usepackage{amsfonts}
\usepackage{fancyhdr}
\usepackage{multicol}
\usepackage{graphicx}

\pagestyle{fancy}
\renewcommand{\headrulewidth}{0pt}
\fancyhead{}
\fancyfoot{}
\rhead{\thepage}

\linespread{2}

\begin{document}
\noindent Christopher Wetherill\\
TBMH 5004\\
Take-Home Quiz 1

\begin{enumerate}
	\item \textit{If the incidence of a disease is increasing but the prevalence is decreasing, how might that be interpreted?}
	
	Incidence of a disease is a measure of new acquisitions in a given time period wheres prevalence is a measure of the number infected at a given point in time. If prevalence is low but incidence high, that might suggest a disease that's common and easy to acquire, but also easy to cure or short-lived in the body. (E.g., the common cold: most people get it at least once a year, but at any given point, only a limited number of individuals are infected)
	
	\item \textit{What differentiates T0 from T1 translational research?}
	
	T0 research focuses, at a very basic level, at identifying problems and novel approaches to tackle them; T1 research involves the translation of that basic T0 research into some product that can be evaluated for clinical effect and applicability.

	\item \textit{List and defend a few of the key strengths and weaknesses of using mouse models for investigating mechanisms and for developing therapeutics for fragile X syndrome.}
	
	One of the most obvious strengths of using mouse models to investigate FX syndrome and develop new therapeutics for it is that researchers are able to perform procedures and collect data that could not easily be gained from human subjects due to regulatory restrictions. The ability to knock in and knock out individual genes, to examine, e.g.,  enhanced mGluR-LTD in hippocampal slices, etc., allows investigation of disorders to a very fine-grained level of control that is simply not practical in human models.
	
	That said, however, an animal model may not always provide a 1:1 analogue to a disease or disorder in humans. Signaling pathways may differ; important physiological structures may differ; other subtle differences may exist that yield ultimately differing results when applied to a human subject. In the specific case of FX syndrome, however, the extant research has largely not identified this to be a significant issue.

	\item \textit{Why is the concept of a polysome (polyribosome) and local protein synthesis important to the mGluR model of fragile x syndrome?}
	
	FMRP is typically synthesized following the activation of Gp1 mGluRs. However, in FX syndrome, the I304N mutation (Bear, Huber, \& Warren, 2004) disrupts the association of FMRP with polyribosomes as an mRNP, resulting in a failure to synthesize proteins needed to halt long-term depression. (I.e., the functional opposition between FMRP and mGluR5 is disrupted by the mutation of FMRP at this RNA-binding domain, resulting in the inability of FMRP to downregulate the activity of mGluR5, and, ultimately, the loss of synaptic AMPA and NMDA receptors.)

	\item \textit{How does an immune inflammatory response occur inside the eye (in the retina) if the eye is a privileged immune site?}
	
	A privileged immune system does not mean ``no immune response'': it only means that these areas are able to suppress an inflammatory response to the introduction of antigens. The eye still has macrophages and dendritic cells that \textit{can} elicit an inflammatory immune response. As noted by MacLachlan, et al. (2010), ``several studies utilizing gene therapy vectors and/or protein administration to the eye have also observed inflammation to varying degrees,'' much as they did with the injection of the AAV2 capsid.

	\item \textit{How would you determine if the pathological processes of age related macular degeneration vs. the loss of central visual acuity is causal in the etiology of depression?}
	
	A simple place to start would be to identify both in what proportion of patients with AMD (and no other chronic illnesses or prior history of mental illness) depression is comorbid and perform a simple statistical analysis (e.g., a $\chi ^2$ test) to assess whether the proportion of depressed AMD patients differs significantly from the general population, populations with other diseases deleterious to vision, or populations with other chronic illness. If no differences exist, there is no obvious covariation between the pathological processes of AMD and the emergence of depression and no causal claims can be made.
	
	If, however, there is covariation and temporal precedence has already been established (i.e., the onset of depression follows the onset of AMD symptoms), then nonspuriousness must be demonstrated. In this case, one could replicate the effects of AMD in, e.g., a mouse model, without the same underlying pathology. If depression emerges in the models at a comparable rate to that seen in AMD patients, then it would be more probable that the loss of central vision is the cause of depression in these cases, and not the pathological processes of AMD.

	\item \textit{What was the standard course of treatment for hepatitis before the advent of anti-virals and why were they generally inadequate?}
	
	Prior to direct-acting HCV antivirals, a regiment of interferons and/or ribavirin was typically used in the treatment of the virus. However, in the course of these treatments, the majority of patients develop severe flu-like symptoms that significantly reduce likelihood of successful treatment completion. Moreover, these treatments only proved mildly efficacious, resolving between 6\% (standard interferon ca. 1991) to 55\% (pegylated interferon ca. 2001; S. McDonald, class presentation, August 27, 2014).

	\item \textit{What factors contribute to the spread of hepatitis C globally, and what have been the barriers to finding effective treatments?}
	
	The perhaps single largest barrier to developing treatments for HCV has been the lack of adequate animal models: to date, chimp models have been the only ones successful and the U.S. has a moratorium on chimpanzee experimentation. Globally, the spread of HCV is attributable largely to preventable causes including needle sharing, unprotected sex, and transfusion of unscreened blood.
\end{enumerate}

\end{document}