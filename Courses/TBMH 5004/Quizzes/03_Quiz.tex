\documentclass[11pt,final] {article}
\usepackage[margin=.95in]{geometry}
\usepackage{setspace}
\usepackage{amsmath}
\usepackage{amsfonts}
\usepackage{fancyhdr}
\usepackage{hyperref}

\pagestyle{fancy}
\renewcommand{\headrulewidth}{0pt}
\fancyhead[REO]{\thepage}
\fancyhead[LEO]{Wetherill}
\fancyfoot{}

\linespread{2}

\begin{document}
\noindent Christopher Wetherill\\
TBMH 5004 Week 3 Quiz

\begin{enumerate}
	\item {\bfseries Can the trajectory and outcomes for a child diagnosed with autism be altered by behavioral intervention? If so, what might be the underlying biological mechanism of the alteration. If not, why not?}
	
	Universally, we don't know. However, in certain instances, there is evidence that early behavioral therapy has been associated with decreased autistic symptomatology and even a reversal of the autistic diagnosis. In these cases, it may be possible that, given that the intervention is in early childhood, the improvements seen are the result of changes to the functional neuronal connectivity of the child's brain (cf. the underconnectivity theory of autism; M. Friedlander, Class Presentation, Sept. 10 2014).
	
	\item {\bfseries Argue for autism being either a single disorder, a continuous spectrum of disorders or a common name for multiple disorders that are distinct.}
	
	Given the evidence that we have seen, I don't know that there is convincing evidence to claim that autism is definitively either a continuous spectrum of disorders or a common name for multiple distinct disorders. Regardless, given the number of genetic mutations implicated in autism (both through partial and whole-genome sequencing), the varying degrees to which the disorder manifests, and the ability of a small subset of autistic individuals to ``mature out'' of the disorder, it seems unlikely that it is a single disorder.
	
	Importantly, though, the frequent co-occurrence of numerous other disorders (e.g., FX syndrome) with autism, without having observed a direct link between autism and any of these disorders, may suggest that autism is in fact a number of separate disorders that are able to manifest the same symptomatology.
	
	\item {\bfseries Describe why mIPSCs that occur in a neuron from a fetal brain would make the neuron more likely to generate action potentials and how oxytocin would affect this process.}
	
	In neonates, GABA, typically a strongly inhibitory neurotransmitter, acts in a primarily excitatory role as a function of the chloride ion concentration gradient (which effectively reverses after birth, giving GABA a then inhibitory effect). As such, the spontaneous typically-inhibitory activity in the neonate would actually translate to an excitatory effect given the excitatory effect of GABA at this state of development. Moreover, in autistic models, it has been shown by Tyzio et al. (2014) that this excitatory/inhibitory switch never occurs at birth, resulting in long-term excitatory GABA activity. These authors further showed that the introduction of oxytocin immediately prior to and during delivery restores normal functioning at birth and specifically restores GABA's excitatory/inhibitory shift.
	
	\item {\bfseries Compare and contrast autism as a neurodevelopmental disorder vs. a disorder that can be switched on and off.}
	
	I think the crux of this question rests on the root cause of autism and whether there is a single cause or a multitude that all manifest the same symptomatology. By some theories of autism as a neurodevelopmental disorder, such as the underconnectivity theory and the theory of oxytocin-mediated GABA inhibition, there may not be strong therapeutic results as things such as neuroplasticity attenuate early in an individual's development. Contrasting, however, if we think of autism as being rooted in something such as a mutation in the SCN2A gene or restricted effect of neurexins during post-natal development, then a therapy that either restores or replicates the missing or impaired neural functionality would likely mitigate the autistic phenotype
	
	\item {\bfseries What other tissues is SCN2A highly expressed (Expression part of GeneCards page)? And, given that it is expressed in these tissues, what would that mean for those tissues if this gene were defective?}
	
	SCN2A protein expression is seen in the plasma, platelets, liver, kidney, heart, and brain. Given this, if this gene were defective, there would likely be significant impairment of activity in these organs. Specifically, the decreased activity of sodium channels in these tissues would likely impair the propagation of action potentials (e.g., in cardiac muscle) as both the passive transport of sodium into the cell and active transport out of the cell following depolarization would be significantly impacted by partially or non-functioning ion channels. Further, research (cf. \url{http://omim.org/entry/182390}) has indicated that mutations in this gene are linked to seizure disorders.
	
	\item {\bfseries If autism patients have a defective SCN2A gene, thus potentially affecting their sodium channel function, which drug (available from Tocris Biosciences, Drugs \& Compounds section of the web page) would you consider for possibly treating these patients?}
	
	Phrixotoxin 3 may be a good choice from among those drugs listed by Tocris given that it specifically targets NA$_V$1.2 channels (with which the SCN2A gene is most associated). The remaining drugs listed, in contrast, are all nonspecific to the sodium channel subtype. Moreover, this drug serves to block inward flow of sodium ions, thereby reducing the hyper-excitability of neurons that is characteristic of mutations in the SCN2A gene.
	
	\item {\bfseries Argue against a deficit in early visual processing in subjects diagnosed with autism, pointing out specific experiments that you would do to show this.}
	
	One could perform an EEG, fMRI, or EEG/fMRI (if you're lucky enough to have those facilities) experiment using both developmentally normal and autistic children in which various visual stimuli are presented to the subjects and correlate neural activity to the presentation of those stimuli. If autism truly gave rise to a deficit in early visual processing, then the autistic subjects should show decreased responsiveness to all stimuli. However, if it (as now seems more probable) is a deficit in the processing of facial cues and emotions, then novel stimuli should still produce an EEG peak at the N400 component (typically associated with surprise) in both autistic and control subjects; however, differences would more likely then emerge in the activation of the fusiform facial area. The neural correlates of simple object recognition should be congruent across all subjects (indicating no visual processing deficits) whereas the recognition and interpretation of facial and emotional cues may still differ (indicating deficits in understanding social constructs, but not in the visual pathway itself, per se).
	
	\item {\bfseries Why do some people think that MMR vaccines cause autism and what would you do to educate them that it is unlikely to be the case?}
	
	The anti-vaxer hype began with a now-refuted article by a British researcher in the late 1990s purporting to have discovered a link between the administration of MMR vaccines with the onset of autism. Despite not being reproducible, using fabricated data, and likely being the product of financial conflicts of interest, a vocal subset of the public seized upon the study as irrefutable fact and rallied around its findings. In cases such as this, when the weight of the evidence clearly does not support the conclusions that the public has come to, I'm skeptical that a program of public education would be particularly effective. (For instance, see the recent congressional subcommittee on Horticulture, Research, Biotechnology, and Foreign Agriculture hearing on GMOs that concluded that the public is so grossly misinformed on the topic that they have no right to weigh in on the science.) Rather, it would seem that positive change could be more readily effected through legislation increasing the barriers and burden of proof required to be granted an exemption from vaccinations.
\end{enumerate}

\end{document}