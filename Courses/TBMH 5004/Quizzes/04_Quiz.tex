\documentclass[11pt,final] {article}
\usepackage[margin=1in]{geometry}
\usepackage{setspace}
\usepackage{amsmath}
\usepackage{amsfonts}
\usepackage{fancyhdr}
\usepackage{multicol}
\usepackage{graphicx}

\linespread{2}

\pagestyle{fancy}
\renewcommand{\headrulewidth}{0pt}
\fancyhead{}
\fancyfoot{}
\rhead{\thepage}

\begin{document}
\noindent Christopher Wetherill \\
TBMH 5004 Week 4 Evaluation\\[0.5cm]

Rotavirus is a genus of dsRNA virus in the \textit{Reoviridae} family known to cause severe, watery diarrhea often accompanied with vomiting, fever, and abdominal pains \cite{Maldonado:1990}. Globally, the virus is attributed to upwards of $525000$ deaths among children under the age of 5 years, with $85\%$ of those deaths in lower-income nations \cite{WHO:2008, Tate:2012}. Although recently approved rotaviral vaccines have shown strong results in developed countries \cite{Giaquinto:2011}, they have low efficacy and death rates remain high \cite{Jiang:2010}. Clearly, vaccines must be developed that are capable of effectively combating the virus even in otherwise immunocompromised populations.

We propose the investigation of siRNA-based vaccines to combat this global disease. Recent research has shown that siRNA-based treatments can be potent inhibitors of viral replication \cite{Barik:2010, Geisbert:2010}. However, this remains a largely novel approach and has to date relied on synthetic siRNAs. In contrast, we seek to use an existing bank of $4000$ locally-maintained siRNAs in a proof-of-concept study that will identify readily-available siRNAs that effectively halt rotavirus replication by promoting the degradation of viral RNA crucial for protein translation. We will then validate these preliminary results to confirm the siRNAs that halt viral replication.

As a direct result of these experiments, we will identify multiple potential therapies that are effective at combating rotavirus infection. Further, although these siRNA may be used individually to combat rotavirus infection, further research may examine the efficacy of cocktails of these compounds which target multiple viral proteins and thus provide both increased efficacy and increased resistance of any therapies against viral mutation and evolution.

Although this research will not take a drug to market, it will seek to provide multiple targets for new and innovative therapies that both are safe to the patient and provide maximal long-term protection against rotavirus infection. Given this, we can expect the research, carried through to market, to provide cheap and effective solutions to a virus that causes half a million deaths annually and serious complications in nearly 2 million more children and accounts for $37\%$ of all diarrheal deaths and $5\%$ of all deaths among children younger than $5$ annually.

\clearpage
\bibliographystyle{plain}
\bibliography{04_references}

\end{document}