\documentclass[11pt,letterpaper,final] {article}

%%%%%%%%%%%%%%%%%%%%%%%%%%%%%%
% Packages
%%%%%%%%%%%%%%%%%%%%%%%%%%%%%%

	\usepackage[margin=1in]{geometry}
	\usepackage{amsmath}
	\usepackage{amsfonts}
	\usepackage{fancyhdr}
	\usepackage{graphicx}
	% \usepackage{tikz}
	% \usepackage{setspace}
	% \usepackage{multicol}

%%%%%%%%%%%%%%%%%%%%%%%%%%%%%%
% Page styling
%%%%%%%%%%%%%%%%%%%%%%%%%%%%%%
	
	%%%%%%%%%%%%%%%%%%%%
	% Headers and footers
	%%%%%%%%%%%%%%%%%%%%
	
	\pagestyle{fancy}
	\renewcommand{\headrulewidth}{0pt}
	\fancyhead{}
	\fancyfoot{}
	\rhead{\thepage}
	
	%%%%%%%%%%%%%%%%%%%%
	% Graphics path
	%%%%%%%%%%%%%%%%%%%%
	
	%\graphicspath{{./assets/}}
	
	%%%%%%%%%%%%%%%%%%%%
	% Frontmatter
	%%%%%%%%%%%%%%%%%%%%
	
	% \title{The Title}
	% \author{The Author}
	% \date{\today}
	
	

%%%%%%%%%%%%%%%%%%%%%%%%%%%%%%
% Custom definitions
%%%%%%%%%%%%%%%%%%%%%%%%%%%%%%
	% Easy scientific notation
	\newcommand{\e}[1]{\ensuremath{\times 10^{#1}}}
	
	% Textual subscripts
	\newcommand{\sub}[1]{\ensuremath{_{\text{#1}}}}
	
	% Textual superscripts
	\newcommand{\super}[1]{\ensuremath{^{\text{#1}}}}

\begin{document}
% \maketitle

\noindent Christopher Wetherill\\
TBMH 5004 --- Week 12, Session 3

\begin{enumerate}
	\item Identify three normal changes in memory that are likely to occur with aging versus disease-related changes.
	
	Disease-related changes may include inability to learn new things; not recalling the names of loved ones; and forgetting how to carry out daily tasks. Typical age-related memory changes may include slight decrements to episodic memory; slower information processing; and decreased ability to multitask and rapidly switch between contexts and tasks.
	
	\item Describe two physiological and three behavioral signs or symptoms of memory impairment.
	
	Ventricles enlarge and the cortex and hippocampus shrink; individuals develop increasing learning and memory impairment, decreased recall vocabulary, and inability to perform complex motor sequences.

	\item Discuss similarities and differences in the ways in which a person with early memory impairment and their care partner talk about and experience memory loss in their everyday lives.
	
	Presumably the caregiver does not experience does not experience abnormal memory loss in his or her everyday life. Beyond this, however, the question is sufficiently broad that it seems unlikely that the range of human responses to disease-related memory changes can be faithfully reduced to a single `way' of discussing such changes by either group of individuals.
	
	\item Briefly share your thoughts about genetic testing for Alzheimer's disease.
	
	Individuals are generally unqualified to interpret the results of their own genetic sequencing. Moreover, analysis of an individual's genetic sequence typically is restricted to exclusively protein-coding regions of the genome which, as increasing evidence shows, fails to provide a full and adequate picture of the genetic factors contributing to the onset of a disease relative to whole-genome sequencing and analysis.
	
\end{enumerate}

\end{document}