\documentclass[11pt,final] {article}
\usepackage[margin=1in]{geometry}
\usepackage{setspace}
\usepackage{amsmath}
\usepackage{amsfonts}
\usepackage{fancyhdr}
\usepackage{multicol}
\usepackage{graphicx}

\pagestyle{fancy}
\renewcommand{\headrulewidth}{0pt}
\fancyhead{}
\fancyfoot{}
\rhead{\thepage}

% \title{The Title}
% \author{The Author}
% \date{\today}

\begin{document}
\noindent Christopher Wetherill\\
Week 10 --- Allergy and asthma

\begin{center}
	Day 1 --- Lawrence
\end{center}

\begin{enumerate}
	\item How do people become sensitized to allergens? (250 words or less, original illustrations welcome).
	
	In allergic sensitization, antigen-presenting cells induce a response from TH2 lymphocytes. These interact with B cells to produce immunoglobulin E. This, finally, leads to an allergic inflammatory response.
		
	\item Besides mast cells, pick one other cell type that you feel plays a very important role in either the sensitization process or chronic inflammatory processes in the context of allergic airway disease and defend your choice. (100 words or less, original illustrations welcome).
	
	B cells also play an important role in allergic responses: specifically, these cells encounter an antigen for which they have a receptor and engulf that antigen. They then cut up the antigen and display its fragments to attract and activate a matching T cell, which then secretes cytokines to mature the B cell and help it proliferate. The B cell finally produces and releases antibodies which bind to the antigen and allow it to be cleared from the body.
	
	\item Besides antihistamines, leukotriene inhibitors or anti-IgE, what might be another immunological target for development of new therapeutic interventions for treatment of allergic disorders. Explain your choice (250 words or less).
	
	The development of epigenetic pharmaceuticals that target genes implicated in immune or inflammatory responses (Gata3, Kitl) or in tissue remodeling (Igf1, Tgfbr1), on the basis of recent research by Collison, et al. could provide new therapeutics for allergic disorders. Specifically, their research identified that a number of microRNAs which target these genes were significantly up- or downregulated following sensitization to an allergen. By inhibiting these epigenetic changes, it may be possible to moderate the sensitization process and prevent the development of severe allergic response.
	
\end{enumerate}

\clearpage

\begin{center}
	Day 2 --- Ie
\end{center}

\begin{enumerate}
	\item What do you feel was the most intriguing and/or surprising aspects of Dr. Ie’s lecture and the patient case presentation in the clinical context of asthma? (50 words minimum).
	
	It was a little surprising that a medical resident who was fully aware of the pathology and etiology of asthma and the effects of leaving it unmanaged left his own asthma unmanaged and uncontrolled for such a long time and had such a cavalier attitude towards it. His attitude seemed more like, ``Whatever happens will happen and there's nothing I can do about it'' rather than, ``This is a chronic condition, but it's completely manageable, so I'm going to do everything I can to keep it controlled and from interfering with my life.''
	
\end{enumerate}

\clearpage

\begin{center}
	Day 3 --- Ramey
\end{center}

\begin{enumerate}
	\item Write a paragraph on some possible contributors to the SES gradient concerning asthma.
	
	The CDC identifies environmental factors such as biologic allergens, environmental tobacco smoke, and irritant chemicals and fumes as significant contributors to the onset of childhood asthma. Growing up in a lower socioeconomic state, a child is significantly more likely to have considerable exposure to these sensitizing agents, above and beyond someone from a higher SES. Lower SES households are more likely to contain smokers; more likely to be exposed to biologic allergens such as mold, dust mites, etc.; and jobs involving the heavy use of harsh cleaning chemicals are predominantly concentrated among these populations. Likewise, low SES households are less likely to seek medical care and to maintain long-term treatment.
\end{enumerate}

\clearpage

\begin{center}
	Day 4 --- Maxwell
\end{center}

\begin{enumerate}
	\item Determine an ``asthma-related'' question (hypothesis) that you would like to research and provide me that question (one sentence only).
	
	Can whole-genome sequencing and analysis provide insight into the genetic predisposition for the development of asthma, and to levels of specificity and sensitivity significantly greater than exome-only sequencing is able to provide?
	
	\item Pick a model to test your hypothesis. It can be an animal model or some other model - even something that you find on-line. Provide a general description of the model and briefly defend your choice - no more than 250 words total (and illustrations are also welcome)
	
	A human model would be appropriate for this study, given that there already exist a large number of human genome repositories from which to draw and that samples from asthmatic populations could likely be readily obtained if necessary.
	
	\item List three or four the biomarkers that you would like to use
	
	Based off of prior research (Kong, Q., Li, W., Huang, H., \& Fang, J. (2014). Differential Gene Expression Profiles of Peripheral Blood Mononuclear Cells in Childhood Asthma. \textit{Journal of Asthma}.), we might choose to focus on ADAM33, Smad7 and LIGHT.
	
	\item List two strengths and two weaknesses of your model
	
	The study would be both relatively easy to conduct given the increasing amount of genetic information available to researchers and would provide a detailed profile to identify children who are predisposed to developing asthma. However, the study is also unable to directly provide or lead to the development of therapeutics for asthmatics, nor is it able to account for the environmental factors that ultimately lead to the development of asthma.
	
\end{enumerate}
\end{document}