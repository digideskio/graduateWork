\documentclass[11pt,final,twocolumn,twoside] {article}
\usepackage[margin=1in]{geometry}
\usepackage{amsmath}
\usepackage{fancyhdr}
\usepackage{lettrine}

\pagestyle{fancy}
\renewcommand{\headrulewidth}{1pt}
\fancyhead{}
\fancyfoot{}
\fancyhead[CEO]{\itshape Editorials}
\fancyfoot[LE,RO]{\bfseries \thepage}
\fancyfoot[RE,LO]{J Trans Biol, Med, \& Health, Vol. 23, No. 3 $\cdot$ Oct 21, 2014}

\setcounter{page}{116}
\setlength{\columnseprule}{0.25pt}

\begin{document}
{\Large \noindent Learning from CAST: Lessons for research in infectious disease}\\

\noindent \lettrine[nindent=0em,lines=2]{A}mong survivors of myocardial infarction, ventricular premature depolarizations represent a significant risk factor for patient death. In this issue of the \textit{Journal}, the Cardiac Arrhythmia Suppression Trial (CAST) staff report preliminary results of a multi-center, randomized, placebo-controlled study designed to ``test whether the suppression of asymptomatic or mildly symptomatic ventricular arrhythmias after myocardial infarction would reduct the rate of death from arrhythmia.''\cite{CAST}

Indeed, the authors concluded that, contrary to prevailing medical practice, the administration of encainide and flecainide to patients who had previously suffered from myocardial infarction increased risk of death relative to the placebo control group to a level that achieved statistical significance. Clearly, this is both a novel and troubling discovery to the fields of medicine and cardiovascular research. However, the lessons learned from this study can be readily applied to the field of immunology and infectious disease.

Specifically, we see this as an occasion to revisit both how we model infectious agents {\itshape in vitro} and in small animal models and how we approach human trials of anti-infectious therapeutics to avoid results similar to those obtained here.

\subsection*{The need for adequate models of disease}

In {\itshape Conjectures and Refutations}, philosopher of science Karl Popper argued that for any theory to be scientifically valid it must risk utter refutation, and that any experiment seeking to examine such a theory must be ``an attempt to falsify it, or to refute it. Testability is falsifiability.''\cite{Popper} And indeed, when investigating possible therapeutics to any infectious agent, we as scientists must bear this in mind and investigate models of infection that adequately reflect this need.

To this end, we see one possible failure in the CAST trial in that prior to it, there had not been investigation of encainide and flecainide in whole-heart models (though admittedly due to technological constraints). That is, we had seen confirmatory evidence in a limited number of models and then extrapolated these successes to clinical practice. Little thought was given to designing experiments that attempted to disprove the use of these drugs (for example, with chronic administration) as therapeutic. Given this, although mechanistic studies were able to elucidate many of the mechanisms of the drugs' action, they failed to consider the effects in a whole, living organism and instead took a limited number of acute successes as indicators for broader applicability.\cite{Soyka}

In these cases, a purely mechanistic model at the level of the cell (or, generally, at the sub-organismal level) may often be inadequate. Imagine, for instance, a novel therapy that effectively halted replication of {\itshape Clostridium difficile} colonies on a blood agar plate, but that, in clinical trials, is found to only be effective in doses that are toxic to a subset of patients who are poor metabolizers of said drug.

Unfortunately, this is not a too-far-fetched hypothetical: in response to the CAST study, a number of pharmacologists suggested in letters to the editor that the effects observed may in fact have resulted from such poor metabolization of the drug in a small subset of the experimental population.\cite{CAST_Review} (Although this reasoning has been largely unsubstantiated with respect to the CAST trial, it remains a legitimate concern and a call to synthesize knowledge of sub-cellular, cellular, and animal models of pathogens and pathogen-host interactions.)

What is ultimately important here is that, keeping in mind Popper's urgings, we thoroughly test anti-infectious agents against the potential for adverse effects in all applicable contexts and use cases prior to embarking on larger clinical trials; it is the failure to do so that more often leads to unintended negative consequences as were seen in CAST. Rather than taking confirmatory anecdote as rationale for a large-scale clinical trial, we must instead rely on a thorough understanding of the mechanisms of action of the drug and its interactions with both the pathogen and the host.

\subsection*{Improved study design in human trials}

To our second point, considered study design in human trials is essential to safe outcomes and valid and generalizable data. In the CAST study, the authors made several design decisions that may have contributed to the results observed and thrown into question their validity.

Foremost, the authors selected as participants only high responders to acute flecainide or ecainide treatment. In the treatment of infectious disease, this would be akin to going to a developing nation where rotavirus therapies achieve only about a $40\%$ recovery rate.\cite{Jiang} Here, we would be taking the best responders of an overall immunocompromised population and examining the effect of a therapy in their recovery. Understandably, this fails to provide any semblance of a representative sample, nor is it likely to help decrease the half million annual deaths from the virus.\cite{Tate,WHO}

Unfortunately, in the discipline, it is all too common for therapies to be only minimally effective.\cite{Checkley} When it is often required to maintain a readily-available stockpile of vaccine in the event of outbreak or seasonal trends in infection, and when long-term sustainability of treatment is so paramount, the costs of a partially-effective drug can be enormous.\cite{Halloran}

In any population study that seeks to address the effectiveness of a therapy, the non-responding population is as absolutely crucial as the responding: clearly, a drug that achieves a perfect cure in the population that responds to its administration is not worth taking to market if only a small proportion of the population responds to it at all. By selecting only the $75\%$ of participants who showed initial (acute) response to the drugs, the authors of the CAST trial produced data that were inherently skewed from the beginning.

At no point did they seek to address exactly why a quarter of the population fails to respond to acute treatment with the drugs, nor if there are significant differences in response to chronic administration of the drugs between the two subpopulations.

Also largely missing from this report was a thorough examination of the demographic characteristics that contributed to increased participant mortality following administration of flecainide and ecainide. As reported, there is not adequate information to determine whether the increased risk of death was uniformly distributed across all subpopulations in the experimental group, or if it was concentrated among those with certain characteristics.

Certainly, in the study of disease, considerations such as these are of clinical importance: an understanding of the characteristics that increase or inhibit a drug's efficacy can only lead to improved patient outcomes and more intelligent and guided drug design and improvement.

\subsection*{Concluding remarks}

In clinical trials, results such as those described by the CAST study are unavoidable. Despite the best experimental designs and prior considerations and safeguards, there will always be times when a drug unexpectedly interacts with the host, with the pathogen, or with another drug. As researchers, we can only work to mitigate these adverse outcomes by thorough and careful mechanistic study of of these factors in non-human models prior to clinical administration.

Yet, to do this, the importance of adequate methods to model such complex interactions cannot be understated. Drug discovery and development is necessarily an iterative process, and it falls to researchers to ensure that we are constantly seeking out shortcomings in our current methodologies and ways to otherwise improve our ability to predict human outcomes.

Science often works best when conducted interdisciplinarily; as evidenced by the CAST trial, when such complex systems are in play, a failure to balance specialist expertise with generalist knowledge can lead to severely detrimental outcomes. In the fields of virology and immunology such concerns still apply: to knock out a host cell protein may be to halt viral proliferation, but it may also disrupt a crucial signaling pathway that results in long-term deleterious effects for the patient.

\begin{flushright}
\textsc{Christopher Wetherill}\\
{\footnotesize Virginia Polytechnic Institute and State University}\\
{\footnotesize Roanoke, VA 24016}
\end{flushright}

\bibliographystyle{acm}
\bibliography{08_references}
\rule{3.2in}{1pt}
{\Large \noindent Toward a human vaccines project}\\

\noindent \lettrine[nindent=0em,lines=2]{V}accines are among the most effective interventions in the history of public health, having led to the eradication of smallpox, the near eradication of polio and the prevention of substantial morbidity and mortality from infectious diseases.[1] Today, a large number of vaccines have been licensed for the prevention of bacterial and viral diseases, including virus-induced liver cancers and cervical cancers caused by hepatitis B virus and human papilloma virus, respectively.[2] However, for several diseases for which vaccines are urgently needed, strategies that have proven successful in the past are unlikely to succeed. This is due to vaccinology challenges common to various pathogens in addition to pathogen-specific issues. For example, genetic variation presents considerable challenges for the development of vaccines against influenza virus, human immunodeficiency virus (HIV), hepatitis C virus, blood-stage malaria and other pathogens and diseases.[3]

For HIV and other antigenically variable viruses, the immunization strategies required for driving somatic hypermutation and affinity maturation of antibodies remain unclear. Defects $\ldots$

[Excerpted from \textsc{Koff, C., Gust, I., \& Plotkin, S.} Toward a human vaccines project. {\itshape Nature Immunology, 15} (2014), 589-592.]
\end{document}