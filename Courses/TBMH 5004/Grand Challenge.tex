\documentclass[11pt,letterpaper,final] {article}

%%%%%%%%%%%%%%%%%%%%%%%%%%%%%%
% Packages
%%%%%%%%%%%%%%%%%%%%%%%%%%%%%%

	\usepackage[margin=1in]{geometry}
	\usepackage{amsmath}
	\usepackage{amsfonts}
	\usepackage{fancyhdr}
	\usepackage{graphicx}
	% \usepackage{tikz}
	% \usepackage{setspace}
	% \usepackage{multicol}

%%%%%%%%%%%%%%%%%%%%%%%%%%%%%%
% Page styling
%%%%%%%%%%%%%%%%%%%%%%%%%%%%%%
	
	%%%%%%%%%%%%%%%%%%%%
	% Headers and footers
	%%%%%%%%%%%%%%%%%%%%
	
	\pagestyle{fancy}
	\renewcommand{\headrulewidth}{0pt}
	\fancyhead{}
	\fancyfoot{}
	\rhead{\thepage}
	
	%%%%%%%%%%%%%%%%%%%%
	% Graphics path
	%%%%%%%%%%%%%%%%%%%%
	
	%\graphicspath{{./assets/}}
	
	%%%%%%%%%%%%%%%%%%%%
	% Frontmatter
	%%%%%%%%%%%%%%%%%%%%
	
	% \title{The Title}
	% \author{The Author}
	% \date{\today}
	
	

%%%%%%%%%%%%%%%%%%%%%%%%%%%%%%
% Custom definitions
%%%%%%%%%%%%%%%%%%%%%%%%%%%%%%
	% Easy scientific notation
	\newcommand{\e}[1]{\ensuremath{\times 10^{#1}}}
	
	% Textual subscripts
	\newcommand{\sub}[1]{\ensuremath{_{\text{#1}}}}
	
	% Textual superscripts
	\newcommand{\super}[1]{\ensuremath{^{\text{#1}}}}

\begin{document}
% \maketitle

{\bfseries Project Objective:} Identify the hallmarks of the tumor microenvironment\\[0.2cm]

{\bfseries Motivating Factors:}
\begin{enumerate}
	\item We have characterized the ``hallmarks of cancer.'' but no such high-level map exists for the tumor microenvironment
	\item There exist 3 primary models for the study of cancer:
		\begin{enumerate}
			\item Cell culture
			\item Lab animal
			\item Human (clinical)
		\end{enumerate}
	Yet, none of these models is adequate for the study of the tumor microenvironment (i.e., cancer's functional interactome). Namely,
		\begin{enumerate}
			\item Cell cultures compress a tumor into a 2D environment and fail to replicate the full microenvironment with which the tumor interacts
			\item Animal models may lack fidelity with respect to human cancers and findings may be difficult to translate into clinical applications
			\item Human models are limited to clinical trials and lack the flexibility found in cell and animal models.
		\end{enumerate}
	\item A high-fidelity {\itshape ex vivo} model with tightly-controlled microenvironmental parameters that mimic the human tumor environment is needed to fill this gap
		\begin{enumerate}
			\item 3D organ scaffolds may offer a solution to this
			\item Such a solution must be:
				\begin{enumerate}
					\item Easy to produce in batch;
					\item Functional replicates of human systems; and
					\item Tightly controlled with respect to the microenvironment
				\end{enumerate}
			\item I.e., any solution must have all the benefits of a human model without the drawbacks of committing a felony in your research
		\end{enumerate}
\end{enumerate}

{\bfseries Anticipated Challenges:}
\begin{enumerate}
	\item What can be grown? Are we restricted to tissue types? To hollow organs? What are the technological challenges of allowing for growth of multiple tissue types/organs?
	\item How do we achieve functional maturity of the organs once grown?
	\item What types of cancer is this model suited to study? How does this relate back to the types of organs and tissues that we are able to grow?
	\item How detailed can we reasonably make our models without replicating every organ system in the entire body? Where do we draw the line between the model lacking fidelity and being over-engineered?
	\item How can any findings be validated and then translated into new therapeutics?
	\item Will any new therapies actually be worth this initial investment in the technology?
\end{enumerate}

{\bfseries Research Timeline:}

a;lsdfknag?

\end{document}